\chapter{Table}

The schedule \ref{tab:activities} may be subject to changes on short notice. These activities are the \textit{do} stage of the Deming Wheel.

\begin{tabularx}{\linewidth}{lX}
	% Catpion and label
	\caption{List of audit activities to perform}\label{tab:activities} \\
	% Header
	\textbf{Date} & \textbf{Activity} \\ \hline\hline
	% Body
	\makecell[tl]{01-01-2018} & \makecell[tX]{On-site audit activities:} \\
	\\
	\makecell[tl]{09:00} & \makecell[tX]{
		\textbf{Opening Meeting} \\
		Establish personal contact with the auditee, confirm the plan for carrying out the audit, explain and confirm the activities, roles and responsibilities of those involved in the audit, confirm communication arrangements and reporting requirements and provide an opportunity for the auditee to clarify issues and ask any questions.
	}
	\\
	\\
	\makecell[tl]{10:00} & \makecell[tX]{
		\textbf{Documentation} \\
		Documentation about scoping, planning, implementing, monitoring and reviewing.
	}
	\\
	\\
	\makecell[tl]{12:00} & \makecell[tX]{
		\textbf{Information security management system (ISMS)} \\
		This meeting will be held with security officers aswell as the process manager.
	}
	\\
	\\
	\makecell[tl]{21:00} & \makecell[tX]{
		\textbf{Closing Remarks (Audit Follow-Up)} \\
		The primary purpose of this meeting is to present the audit findings and conclusions, ensure a clear understanding of the results, and agree on the timeframe for corrective actions. This meeting can be held at the end of the audit.
	}
	\\ \hline
\end{tabularx}
