\chapter{Übungsblatt 1}

\section{Aufgabe 1}

Welche Betriebsarten werden von Betriebssystemen angeboten und welchen Sinn
haben sie?

\begin{itemize}
    \item \textbf{Echtzeitbetriebssysteme}: Man unterscheidet zwischen harten- und
          weichen Echtzeitbetriebssystemen. Bearbetet \texttt{I/O} in festen Zeitgrenzen.
          Das Betriebssystem darf die Ausführung des Programms nicht verzögern. Es müssen
          alle benötigten Ressourcen verfügbar sein.
          \begin{itemize}
              \item \textbf{Harte Echtzeitbetriebssysteme}: Hat eine fest vorgelegte
                    maximale Antwortzeit. Wird diese überschritten, wirft das Betriebs\-
                    system einen Fehler.
              \item \textbf{Weiche Echtzeitbetriebssysteme}: Antwortzeiten sollten
                    möglichst schnell erscheinen. Verspätete Antwortzeiten sind unerwünscht,
                    aber kein harter Fehler.
          \end{itemize}
    \item \textbf{Batchbetriebssysteme (Stackbetriebssysteme)}: Nutzer erstellt
          Beschreibung seiner Aufgabe(n). Betriebssystem arbeitet der Reihenfolge nach
          einen Stapel von Anweisungen durch. Kommandoprozeduren müssen sämtliche
          \texttt{I/O} Anweisungen enthalten, manchmal organisatorische Angaben.
          \textit{Beispiele}: Backup, Mathematische Probleme, Compilervorgänge.
    \item \textbf{Dialogbetriebssysteme}: Sollen den Anwender/-in schnelle
          Antworten geben und im Dialoig mit ihm/ihr stehen. Das System wartet auf
          Anweisungen und Befehle des Anwenders. Ziel ist minimale Antwortzeit.
    \item \textbf{Hintergrundbetrieb}: Alle Prozesse und Anwendungen laufen
          im Kernel Modus. Anwender hat keinen Einfluss auf laufende Prozesse.
          Zwischenschema zwischen \textit{Dialogbetrieb} und \textit{Batchbetrieb}.
\end{itemize}

\section{Aufgabe 2}

Was ist ein Timesharing-System?

\section{Aufgabe 3}

Erläutern Sie anhand einer Skizze das Prinzip der Speicherverwaltung mittels
"Swapping" und "Paging".

\section{Aufgabe 4}

\begin{enumerate}
    \item Welche Eingabe- und Ausgabemöglichkeiten für Computer kennen Sie?
    \item Beschreiben Sie jeweils die zugehörigen Datentypen und mögliche Besonderheiten
          im Hinblick auf die Aufgaben des Betriebssystems.
\end{enumerate}

\section{Aufgabe 5}

Man skizziere und erläutere den Ablauf eines Systemaufrufs aus einem
Benutzerprogramm heraus.