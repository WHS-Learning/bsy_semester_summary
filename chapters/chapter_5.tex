\chapter{Kapitel 5}

\section{Grundlagen der Ein- /Ausgabehardware}

Man unterscheidet bei Ein- und Ausgabegeräten zwischen Block- und
Zeichenorientierten Geräten. \texttt{I/O} Geräte haben sehr unterschiedliche
Geschwindigkeiten. Von Tastaturen mit Datenraten von 10 Bytes pro Sekunde bis
hin zu SONET OC-768 mit 5GB pro Sekunde.

\subsection{Controller}

\texttt{I/O} Geräte sind in Mechanik und Elektronische Komponenten aufgebaut.
Die elektronischen Komponenten heißen \textbf{Steuerwerk}. Sie verwalten nahezu
identische geräte. In PCs sind sie oft auf der Hauptplatine oder als Steckkarte.
Das Betriebssystem ist immer mit dem Steuerwerk beschäftigt. Die Schnittstelle
zwischen Steuerwerk und Gerät befindet sich auf einer niedrigen Ebene.

\subsection{Memory-Mapped-Ein-/Ausgabe}

Das Steuerwerk besitzt eigene Register zur Kommunikation mit der CPU. Kommandos
werden direkt in Steuerweksregister geschrieben. Es gibt einen Speziellen
Adressbrereich für die \texttt{I/O}, auf den nur das Betriebssystem zugriff
hat. Kontrollregister können auch in den Arbeitsspeicher eingeblendet werden.

\subsection{Direct Memory Access}

\textbf{Arbeitsweise ohne Direct Memory Access:}

\begin{enumerate}
    \item Leseanfrage.
    \item Steuerwerk liest einen gesamten Block Byte für Byte ein.
    \item Prüfsumme wird berechnet, um Lesefehler auszuschließen.
    \item Steuerwerk löst Unterbrechung aus.
    \item Betriebssystem liest Block aus Puffer des Controllers und kopiert ihn
          zeichen-/wortweise in den Hauptspeicher.
\end{enumerate}

\textbf{Arbeitsweise mit Direct Memory Access:}

\begin{enumerate}
    \item Betriebssystem übergibt Blockadresse, Speicheradresse und Anzahl der zu
          lesenden Bytes an den DMA-Controller.
    \item Kommando wird an den Platten-Controller ausgegeben, damit er die Daten von der
          Platte einliest und verifiziert.
    \item Direct Memory Access Controller leitet Datentransfer mit einem Lesebefeh über
          den Bus zum Platten-Controller ein.
    \item Steuerwek schreibt Zeichen direkt in den Hauptspeicher.
    \item Nach dem Schreiben wird vom Platten-Controller an den Direct Memory Access
          Controller ein Fertigsignal gesendet.
    \item Schritt 2 bis 4 werden so lange wiederholt bis alle Daten im Hauptspeicher
          sind.
    \item Wenn alle Daten im Hauptspeicher sind löst Direct Memory Access Controller
          Interrupt aus. Fertigmeldung.
\end{enumerate}

\subsection{Interrupts}

Sobald ein \texttt{I/O} Gerät eine Aufgabe beendet hat, erzeugt es über den
Interrupt-Controller einen Interrupt. Dieser enthält eine Nummer die als Index
für den Interrupt-Vektor dient. Nachdem die Interrupt-Verarbeitung durch die
CPU begonnen wurde, wird der Interrupt beim Interrupt-Controller bestätigt.

\section{Grundlagen der Ein- /Ausgabesoftware}

\texttt{I/O} Geräte sollten von Geräten unabhängig funktionieren. Programme sollten auf unterschiedlichen Geräten gleich funktionieren. Fehler sollten möglichst immer an der Fehlerquelle behandelt werden.

\texttt{I/O} kann über drei Ansätze durchgeführt werden:

\begin{enumerate}
    \item Programmierte \texttt{I/O}
    \item Interruptgesteuerte \texttt{I/O}
    \item \texttt{I/O} mit Direct Memory Access
\end{enumerate}

\subsection{Programmierte \texttt{I/O}}

Das \texttt{I/O} Gerät wird vom Benutzerprozess reserviert. Das Betriebssystem
kopiert dann die auszugebenen Daten in einen Puffer im Kernardressbereich. Die
Daten können dann (einzeln) ins \texttt{I/O} Gerät kopiert werden. Der
Prozessor ist hier komplett belegt, bis die gesamte \texttt{I/O} beendet ist.
Aktives warten ist hier u.u. nötig.

\begin{lstlisting}[language=C]
copy_from_user(buffer, p, count);
for (i = 0; i < count; i++) {
    while (*printer_status_reg != READY);
    *printer_status_reg = p[i];
}
return_to_user();
\end{lstlisting}

\subsection{Interruptgesteuerte \texttt{I/O}}

Wenn auf ein \texttt{I/O} Gerät gewartet werden muss, findet ein Prozesswechsel
statt. Das \texttt{I/O} Gerät wird u.u. erst vom Benutzerprozess reserviert.
Das Betriebssystem kopiert dann die auszugebenen Daten in einen Puffer im
Kernardressbereich. Das Erste Zeichen wird zum \texttt{I/O} Gerät geschickt,
sobald dieses bereit ist. Der Scheduler scheduled einen anderen Prozess. Der
Prozess, der das Drucken der Zeichenkette beauftragt hat, wird so lange
blockiert, bis die gesamte Ausgabe beendet ist.

\subsection{\texttt{I/O} mit Direct Memory Access}

Bei der Interruptgesteuerten \texttt{I/O} werden sehr viele Interrupts erzeugt,
welche viel CPU-Zeit inanspruch nehmen. Der Direct Memory Access Controller
kann die Kopieroperation übernehmen, um CPU-Zeit zu sparen. Es kommt dadurch
nur zu einem Interrupt pro Puffer. Der Direct Memory Access Controller ist
jedoch deutlich langsamer, als die CPU, wodurch es sinnvoll sein kann, die CPU
zu nutzen, wenn \texttt{I/O} per Direct Memory Access Controller zu langsam
ist.

\section{Schichten der Ein- /Ausgabesoftware}

\texttt{I/O} Software wird in vier Schichten aufgeteilt, welche jeweilts zu den angrenzenden Schichten Schnittstellen besitzen. Die Funktionalität und Schnittstellen sind Betriebssystem abhängig.

\begin{enumerate}
    \item \texttt{I/O}-Software der benutzerebene
    \item Geräteunabhängige Betriebssystemsoftware
    \item Gerätetreiber
    \item Unterbrechungsroutinen
\end{enumerate}

\subsection{Unterbrechungsroutinen}

Interrupts sollten möglichs im inneren des Betriebssystems verborgen sein. Der
Treiber, der die \texttt{I/O} gestartet hat, kann blockiert werden, bis die
Operation beendet wird und der entsprechende Interrupt eintritt. Der Treiber
selbst kann sich beispielsweise durch die \texttt{down} Operation blockieren.
Anschließend muss die Unterbrechungsroutine den Treiber mittels \texttt{up}
wieder deblockieren. Dies Funktioniert am besten, wenn der Treiber als
Kernprozess mit eigenem Zustand, Stack und Befehlszähler implementiert ist.

\subsection{Gerätetreiber}

Der Code von Gerätetreibern ist Geräteabhängig. Dieser wird vom Hersteller des
Gerätes für die entsprechenden Betriebssysteme erstellt. Jeder Treiber
verwaltet einen Gerätetyp oder eine Klasse von eng verwandten Geräten. Der
Gerätetreiber schickt Kommandos an die Hardware und überprüft auf korrekte
Ausführung. Der Treiber ist der einzige Teil der Betriebssystem Software der
weiß, wie viele Register mit welcher Funktion ein Gerät besitzt.

\subsection{Geräteunabhängige \texttt{I/O}-Software}

Die Grenze zwischen Treiber und der Geräteunabhängigen Software ist
Betirebssystem abhängig. Es kann eine einheitliche Schnittstelle bereitgestellt
werden, die alle \texttt{I/O}-Funktionen durchführen kann, bei der die Geräte
gleich sind.

\subsection{\texttt{I/O}-Software im Benutzeradressraum}

Ein Kleiner teil der \texttt{I/O}-Software bedindet sich in Bibliotheken für
Benutzeraufrufe. Hierbei wird zwischen \textbf{Low-Leve-Routinen} und
\textbf{High-Level-Routinen} unterschieden. Die LL-Routinen bereiten nur
Parameter für Systemaufrufe auf (z.B. \texttt{write()}). HL-Routinen nehmen
verarbeitung von Daten vor und benutzen LL-Routinen (z.B. \texttt{printf}).
Spooling-Systeme zählen auch zu der \texttt{I/O}-Software.

\section{Plattenspeicher}

\subsection{Strategien zur Steuerung eines Plattenarms}

\subsubsection{First Come First Served (FCFS)}

Falls der Plattentreiber nur eine Anfrage zu einer Zeit bearbeiten kann, werden
diese in Reihenfolge des Eintreffens bearbeitet. Keine Optimierung der
Suchzeit.

\subsubsection{Shortest Seek First (SSF)}

Treiber organisiert Anfragen so, dass der Arm die kürzeste Distanz zurücklegen
muss. Der Kopf wird sich hier in der Regel in der Plattenmitte aufhalten.
Anfragen an die äußeren Zylinder werden benachteiligt.

\subsubsection{Afuzugsalgorithmus (elevator algorithm)}

Der Plattenkopf bewegt sich immer in eine Richtung (hoch oder runter). Die
Richtung wird gewechselt, wenn keine Anfragen in dieser Richtung mehr bestehen.

\section{Uhren}

Uhren haben die Aufgabe, die Aktuelle Urzeit zu speichern, zu verhindern, dass
ein Prozess die CPU nicht freigibt und Alarme zu senden. Heute werden Uhren
durch Quarz Synchronisiert. Früher wurden sie durch Netzfrequenz
Synchronisiert.

\subsection{Funktion der Quarz-Uhr}

Ein Zähler wird mit dem Quarzschwingungen dekrementiert. Ist der Zähler = 0,
erzeugt die Uhr eine Unterbrechung und der Zähler wird aus dem Halte-Register
neu geladen. Halte-Register liefer eine Wählbare Unterbrechungssequenz, welche
auf die Uhrzeit umgerechnet wird.

\subsection{Software Uhren}

Hardware Uhren lösen nach einem bestimmten Zeitintervall einen Interruptaus.
Der rest wird vom Uhrentreiber erledigt. Dieser Verwaltet die Uhrzeit,
Verhindert, dass Prozesse zu lange laufen, Führt Buch über die
Prozessornutzung, behandelt den \texttt{alarm}-Systemaufruf, Stellt Uhren zur
Überwachung von Betriebssystemaktivitäten bereit und monitort und Protilet
Statistiken.