\chapter{Übungsblatt 3}

\section{Aufgabe 1}

Was versteht man unter Relokation und Schutz? Welche Lösungsmöglichkeiten gibt es?

\section{Aufgabe 2}

Wie funktioniert das Verfahren der Speicherverwaltung mit Bitmaps? Wie funktioniert das Verfahren der Speicherverwaltung mit einer verketteten Liste? Vergleichen Sie die Verfahren. Gegeben sei folgende Speicherbelegung:

\includegraphics[width=\textwidth]{assets/uebungsblatt_3_task_2.png}

\section{Aufgabe 3}

Welche Algorithmen zur Speicherallozierung kennen Sie? Diskutieren Sie die Vor- und Nachteile der Verfahren!

\section{Aufgabe 4}

Gegeben ist die folgende Adresse beim 4-stufigen Paging.

\includegraphics[width=\textwidth]{assets/uebungsblatt_3_task_4.png}

Zeichnen Sie ein Beispiel einer passenden Seitenrahmentabelle auf. Wie viele Einträge können die einzelnen Tabellen enthalten? Erklären Sie das Verfahren der Adressierung anhand der Adresse: 0000000000 101 001 10 00 100100100101. Warum hat man mehrstufige Seitenrahmentabellen eingeführt? Welche Probleme bringt dieses Verfahren mit sich? Wie kann man das 4-stufige Verfahren beschleunigen?

\section{Aufgabe 5}

Gegeben sei ein Speicherbereich von einem GByte. Der Reihe nach kommen Anfragen der Größe A=113 MB, B=43 MB, C=386 MB, D=96 MB an.

Zeichnen Sie das Prinzip der Speicherplatzvergabe nach dem Buddy-System anhand der unten vorgegebenen Zeichnung auf.

\begin{tikzpicture}[
  font=\sffamily,
  every node/.style={font=\sffamily}
]

  \def\lineLength{8cm}
  \def\tickLabelOffset{1.5mm}
  \def\ySpacing{1.2cm}
  \def\xLineStart{3cm}
  \def\mainLabelXPos{\xLineStart - 0.5cm}

  \tikzset{
    custom diamond/.tip={Diamond[length=6pt, width=4.5pt, fill=black]}
  }

  \newcommand{\drawScale}[3]{
    \begin{scope}[yshift=-#1*\ySpacing]
      \node[anchor=east] at (\mainLabelXPos, 0) {#2=#3 MB};
      
      \coordinate (lineStart) at (\xLineStart, 0);
      \coordinate (lineEnd) at (\xLineStart + \lineLength, 0);
      
      \draw[-{custom diamond}] (lineStart) -- (lineEnd);
      \draw[{custom diamond}-{custom diamond}] (lineStart) -- (lineEnd);
      
      \node[below=\tickLabelOffset of lineStart, font=\small] {0};
      \node[below=\tickLabelOffset of lineEnd, font=\small] {1024};
    \end{scope}
  }

  \drawScale{0}{A}{113}
  \drawScale{1}{B}{43}
  \drawScale{2}{C}{386}
  \drawScale{3}{D}{96}

\end{tikzpicture}