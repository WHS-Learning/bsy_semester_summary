\chapter{Kapiel 2}

\begin{enumerate}[label=\arabic*.]
      \item \textbf{Wieso spricht man im Kontext von Prozessen von einer `Illusion der Parallelität'?} \newline
            \textit{Antwort:}

      \item \textbf{Wie erhält der Scheduler selbst die Prozessorzeit?} \newline
            \textit{Antwort:}

      \item \textbf{Durch welche Ereignisse wird ein Prozess erzeugt?} \newline
            \textit{Antwort:}

      \item \textbf{Welche Gründe kann es haben, wenn ein Prozess sich beendet?} \newline
            \textit{Antwort:}

      \item \textbf{Was ist ein Zombie-Prozess und wie entsteht so einer?} \newline
            \textit{Antwort:}

      \item \textbf{In welcher Struktur sind Kindprozesse aufgestellt? (UNIX)} \newline
            \textit{Antwort:}

      \item \textbf{Wie werden Prozesse unter WIndows hierarchiert?} \newline
            \textit{Antwort:}

      \item \textbf{Zeichnen Sie ein Zustandsdiagramm, was die Zustände von Prozessen darstellt.
                  (Vergessen Sie nicht die Übergänge zu beschriften!)} \newline
            \textit{Antwort:}

      \item \textbf{Welche Aufgaben muss der Scheduler erfüllen?} \newline
            \textit{Antwort:}

      \item \textbf{Nennen Sie 8 grundlegende Infos die sich in einer Prozesstabelle befinden.} \newline
            \textit{Antwort:}

      \item \textbf{Wo befindet sich der Interupt-Vektor und was genau ist das?} \newline
            \textit{Antwort:}

      \item \textbf{Wie ist der Ablauf eines Interrupts?} \newline
            \textit{Antwort:}

      \item \textbf{Warum werden interaktive Prozesse beim Scheduling priorisiert?} \newline
            \textit{Antwort:}

      \item \textbf{Wie erkennt der Scheduler I/O-intensive Prozesse?} \newline
            \textit{Antwort:}

      \item \textbf{Was sind die Vorteile von mehreren Threads anstelle von mehreren Prozessen?} \newline
            \textit{Antwort:}

      \item \textbf{Was sind die grundlegenden Unterschiede zwischen Prozessen und Threads?} \newline
            \textit{Antwort:}

      \item \textbf{Welche Elemente sind für alle Threads zugreifbar? Welche sind individuell für Threads
                  zugeteilt?} \newline
            \textit{Antwort:}

      \item \textbf{Was versteht man unter dem Begriff `Multi-Threading'?} \newline
            \textit{Antwort:}

      \item \textbf{Was versteht man unter Interprocess Communication?} \newline
            \textit{Antwort:}

      \item \textbf{Was versteht man unter einer Race Condition?} \newline
            \textit{Antwort:}

      \item \textbf{Was ist eine kritische Region?} \newline
            \textit{Antwort:}

      \item \textbf{Zeichnen Sie anhand eines Aktivitätsdiagramm ein Beispiel für das Zusammenspiel zweier
                  Prozesse, die zur selben Zeit ausgeführt werden wollen.} \newline
            \textit{Antwort:}

      \item \textbf{Was genau versteht man unter `Aktives Warten'?} \newline
            \textit{Antwort:}

      \item \textbf{Was ist Polling? Was sind die Vor-/Nachteile?} \newline
            \textit{Antwort:}

      \item \textbf{Was ist das `Prioritäteninversionsproblem'} \newline
            \textit{Antwort:}

      \item \textbf{Wie funktionieren Spinlocks?} \newline
            \textit{Antwort:}

      \item \textbf{Wie funktioniert das strikte Alternieren?} \newline
            \textit{Antwort:}

      \item \textbf{Wie funktioniert das Peterson Verfahren?} \newline
            \textit{Antwort:}

      \item \textbf{Wie funktioniert TSL?} \newline
            \textit{Antwort:}

      \item \textbf{Was genau versteht man unter `Passives Warten'?} \newline
            \textit{Antwort:}

      \item \textbf{Welche zwei Systemaufrufe kennen Sie um Passives Warten umzusetzen?} \newline
            \textit{Antwort:}

      \item \textbf{Nennen Sie zwei Gründe warum passives Warten besser als aktives Warten ist.} \newline
            \textit{Antwort:}

      \item \textbf{Welche Operationen werden bei Semaphoren benutzt und wie funktionieren sie?} \newline
            \textit{Antwort:}

      \item \textbf{Was ist ein Ereigniszähler und wie funktioniert er? (Nenne seine Funktionen und wie sie
                  funktionieren)} \newline
            \textit{Antwort:}

      \item \textbf{Worin liegt der Unterschied zwischen TSL und XCHG?} \newline
            \textit{Antwort:}

      \item \textbf{Wie funktionieren Monitore und wo werden sie hauptsächlich genutzt?} \newline
            \textit{Antwort:}

      \item \textbf{Was versteht man unter einem wechselseitigen Ausschuss?} \newline
            \textit{Antwort:}

      \item \textbf{Wie kann man Prozesskommunikation in verteilten Systemen realisieren?} \newline
            \textit{Antwort:}

      \item \textbf{Wo liegen die Nachteile beim Nachrichtenaustausch?} \newline
            \textit{Antwort:}

      \item \textbf{Erklären sie anhand einer Zeichnung was eine Barriere ist und wo die Vorteile liegen?} \newline
            \textit{Antwort:}

      \item \textbf{Warum sind Prozesswechsel sehr aufwendig? (bzw\. kosten viel Rechenzeit)} \newline
            \textit{Antwort:}

      \item \textbf{Erklären Sie den Unterschied zwischen CPU-bounded und I/O-bounded Scheduling.} \newline
            \textit{Antwort:}

      \item \textbf{Wann schaltet sich der Scheduler ein?} \newline
            \textit{Antwort:}

      \item \textbf{Wo liegt der Unterschied zwischen unterbrechenden und nicht-unterbrechenden
                  Schedulern?} \newline
            \textit{Antwort:}

      \item \textbf{Welche Scheduling-Kategorien kennen Sie? (Nennen Sie die konkreten Ziele der einzelnen
                  Kategorien)} \newline
            \textit{Antwort:}

      \item \textbf{Welche Scheduling-Verfahren kennen Sie? (Erläutern Sie ihre Funktionen)} \newline
            \textit{Antwort:}

      \item \textbf{Was sind periodische und aperiodische Ereignisse?} \newline
            \textit{Antwort:}

      \item \textbf{Erläutern Sie den Unterschied zwischen Threadscheduling auf Benutzerebene und auf
                  Kernebene.} \newline
            \textit{Antwort:}

      \item \textbf{Was genau ist das Philosophen Problem? Und wie kann man es lösen?} \newline
            \textit{Antwort:}

      \item \textbf{Worin unterscheiden sich Petersons Verfahren zum wechselseitigen Ausschluss und TSL?} \newline
            \textit{Antwort:}

      \item \textbf{Bei der Lösung des Erzeuger-Verbraucher-Problems mit Hilfe von Semaphoren werden
                  beim Erzeuger zwei Down-Operationen hintereinander durchgeführt. Was passiert, wenn
                  man die Reihenfolge dieser beiden Operationen vertauscht?} \newline
            \textit{Antwort:}

\end{enumerate}